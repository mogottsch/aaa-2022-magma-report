\clearpage
\section{Problem Description}
\label{sec:problem_description}


In a world with constantly growing greenhouse gas emissions, it has become more important than ever to take action in order to slow down climate change. Road transport alone is responsible for 16\% of man-made CO2 emissions and even despite the technological advances of our modern society, traditional urban mobility still primarily consists of privately owned internal combustion engine vehicles. Besides the acceleration of the greenhouse effect due to air pollution, breathing in these emissions can also be hazardous to the urban population. Moreover, typical private car transport leads to low vehicle utilization. Thus, it is important to explore alternative mobility concepts such as bicycle sharing or e-scooter sharing. These are both examples of the emerging trend of as-a-service or on-demand mobility.

The goal of this research is to use the data provided by the bicycle sharing provider NextBike in the city of Leipzig from the year 2019 to explore how fleet operators can use real-time data to monitor and predict their demand, as well as gain insights in the competitors' business. The first part of research consists of the data collection and data aggregation. Here, we describe how to calculate trips based on raw positional data and how to infer demand and availability of bicycles from it. Additionally, we collect weather, land use and points of interest data. Next, we provide comprehensive spatio-temporal analysis of demand and availability patterns in various resolutions. Our research also includes hard and soft clustering analysis of trips to gain insight into customer behavior. Additionally, the predictive analysis part aims to predict future demand and availability of bicycles across various resolutions and benchmark different machines learning models in this context. Such forecasting has potential to aid in business decisions such as bicycle allocation or determination of the fleet size.